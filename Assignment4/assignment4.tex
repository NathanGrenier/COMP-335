\documentclass[12pt]{article}
\usepackage{graphics}
\usepackage[top=1in,bottom=1in,left=1in,right=1in]{geometry}
\usepackage{alltt}
\usepackage{array}	
\usepackage{graphicx}
\usepackage{tabularx}
\usepackage{verbatim}
\usepackage{setspace}
\usepackage{listings}

\usepackage{amssymb,amsmath, amsthm}
\usepackage{zed-csp}
\usepackage[cc]{titlepic}

\title{COMP 335: Introduction to Theoretical Computer Science\\
\ \\
Assignment 4}
\author{Nathan Grenier}
\date{\today \\ Fall 2024}

\begin{document}
\begin{spacing}{1.5}
      \maketitle

      \newpage

      \begin{enumerate}

            \item[1.] [20 Points] For each of the following languages, give a context-free grammar (CFG).

                  \begin{enumerate}
                        \item[(a)] (5 Points) $L_a = \{a^nb^m : m,n \geq 0 \text{ and } 2n \leq m \leq 3n \}$

                              $S \rightarrow A | \lambda$ \\
                              $A \rightarrow aAbbB | \lambda$ \\
                              $B \rightarrow b | \lambda$

                        \item[(b)] (5 Points) $L_b = \{a^nb^mc^k : k=2m+n \}$

                              $S \rightarrow A $ \\
                              $A \rightarrow aAC_1 | B$ \\
                              $B \rightarrow bBC_2 | \lambda$ \\
                              $C_1 \rightarrow c$ \\
                              $C_2 \rightarrow cc$

                        \item[(c)] (5 Points) $L_c = \{a^nb^mc^k : n=m \text{ or } m \leq k \}$

                              $S \rightarrow X | Y | \lambda$ \\
                              $X \rightarrow A_1C_1$ \\
                              $A_1 \rightarrow aA_1B_1 | \lambda$ \\
                              $B_1 \rightarrow b$ \\
                              $C_1 \rightarrow cC_1 | \lambda$ \\
                              $Y \rightarrow A_2C_2$ \\
                              $C_2 \rightarrow B_2C_2c | \lambda$ \\
                              $B_2 \rightarrow b | \lambda$ \\
                              $A_2 \rightarrow aA_2 | \lambda$

                              \newpage
                        \item[(d)] (5 Points) $L_d = \{w \in \{a,b\}^* : w \neq xx, \text{ for any } x \in \{a,b\}^* \}$

                              $S \rightarrow X | Y$ \\
                              $X \rightarrow aXa | bXb | aXb | bXa | a | b$ \\
                              $Y \rightarrow aYa | bYb | aYb | bYa | aY_1a | bY_2b | aY_3b | bY_4a$ \\
                              $Y_1 \rightarrow bZb | aZb | bZa$ \\
                              $Y_2 \rightarrow aZa | aZb | bZa$ \\
                              $Y_3 \rightarrow aZa | bZb | aZb$ \\
                              $Y_4 \rightarrow aZa | bZb | bZa$ \\
                              $Z \rightarrow aZa | bZb | aZb | bZa | \lambda$

                  \end{enumerate}

                  \newpage
            \item[2.] [10 Points] Consider the language $L=\{a^{n+1}b^n : n \geq 0 \}$

                  \begin{enumerate}
                        \item[(a)] (5 Points) Describe in English the complement $\overline{L}$ of $L$. Your description should specify the types of strings that are in $\overline{L}$. That is, it is not acceptable to say $\overline{L}$ includes every string over $\{a,b \}$ that is not in $L$, which is obviously true.

                              \textbf{Answer:} \\
                              The language $\overline{L}$ contains all possible strings except for those where the number of consecutive $a$'s is one more than the number of consecutive $b$'s, where $a$'s always come before $b$'s. This means that the strings that are in $\overline{L}$ are as follows:
                              \begin{itemize}
                                    \item $\lambda$
                                    \item Strings starting with $b$. ($baab$)
                                    \item Strings ending with $a$. ($aaabba$)
                                    \item Strings where all $a$'s don't come before all $b$'s. ($aabab$)
                                    \item Strings where all consecutive $a$'s come before all consecutive $b$'s and the number of $a$'s is not one more than the number of $b$'s. ($aaabbb$)
                              \end{itemize}

                        \item[(b)] (5 Points) Give a CFG for $\overline{L}$.




                  \end{enumerate}

                  \newpage
            \item[3.] [15 Points] Let $G$ be the following CFG in which $S$ is the start variable:

                  $S \rightarrow AB | aB$ \\
                  $A \rightarrow aab | \lambda$ \\
                  $B \rightarrow bbA$

                  \begin{enumerate}
                        \item[(a)] (5 Points) Using the procedure discussed in the class, convert $G$ into an equivalent grammar in Chomsky Normal Form (CNF).

                              \begin{enumerate}
                                    \item[Step 1:] Remove $\lambda$-productions.

                                          $A \implies \lambda$ \\

                                          $S \rightarrow AB | B | aB$ \\
                                          $A \rightarrow aab$ \\
                                          $B \rightarrow bbA | bb$

                                    \item[Step 2:] Remove unit productions.

                                          $S \implies B$ \\

                                          $S \rightarrow AB | aB | bbA | bb$ \\
                                          $A \rightarrow aab$ \\
                                          $B \rightarrow bbA | bb$

                                    \item[Step 3:] Remove useless productions.

                                          There are none.

                                    \item[Step 4 and 5:] Convert to CNF.

                                          $S \rightarrow AB | T_aB | T_bV_1 | T_bT_b$ \\
                                          $V_1 \rightarrow T_bA$ \\
                                          $A \rightarrow T_aV_2$ \\
                                          $V_2 \rightarrow T_aT_b$ \\
                                          $B \rightarrow T_bV_3 | T_bT_b$ \\
                                          $V_3 \rightarrow T_bA$ \\
                                          $T_a \rightarrow a$ \\
                                          $T_b \rightarrow b$
                              \end{enumerate}

                              \newpage
                        \item[(b)] (5 Points) Find an equivalent grammar to $G$ in Greibach Normal Form (GNF).

                              \textbf{Answer:} \\
                              We can start the conversion from the CNF grammar we found in part (a).

                              \begin{enumerate}
                                    \item[Step 1:] Make sure every production's RHS starts with a terminal. If not, substitute the variable until the first character is a terminal (or a variable that encodes a terminal).

                                          $S \rightarrow T_aV_2B | T_aB | T_bV_1 | T_bT_b$ \\
                                          $V_1 \rightarrow T_bA$ \\
                                          $A \rightarrow T_aV_2$ \\
                                          $V_2 \rightarrow T_aT_b$ \\
                                          $B \rightarrow T_bV_3 | T_bT_b$ \\
                                          $V_3 \rightarrow T_bA$ \\
                                          $T_a \rightarrow a$ \\
                                          $T_b \rightarrow b$

                                    \item[Step 2:] Substitute the variables that encode terminals that are at the start of the RHS of the productions.

                                          $S \rightarrow aV_2B | aB | bV_1 | bT_b$ \\
                                          $V_1 \rightarrow bA$ \\
                                          $A \rightarrow aV_2$ \\
                                          $V_2 \rightarrow aT_b$ \\
                                          $B \rightarrow bV_3 | bT_b$ \\
                                          $V_3 \rightarrow bA$ \\
                                          $T_a \rightarrow a$ \\
                                          $T_b \rightarrow b$
                              \end{enumerate}

                              \newpage
                        \item[(c)] (5 Points) Suppose we modify the original grammar $G$ as follows: remove the $\lambda \text{-production}$ $A \rightarrow \lambda$ and instead add the unit production $A \rightarrow A$. Let us call the resulting grammar $G'$. Convert $G'$ into CNF, and simplify, if possible. Also describe in English language $L(G')$.

                              $G'= $ \\
                              $S \rightarrow AB | aB$ \\
                              $A \rightarrow aab | A$ \\
                              $B \rightarrow bbA$

                              \begin{enumerate}
                                    \item[Step 1:] Remove $\lambda$-productions.

                                          There are none

                                    \item[Step 2:] Remove unit productions.

                                          $A \implies A$ can be removed immediately.

                                          $S \rightarrow AB | aB$ \\
                                          $A \rightarrow aab$ \\
                                          $B \rightarrow bbA$

                                    \item[Step 3:] Remove useless productions.

                                          There are none.

                                    \item[Step 4 and 5:] Convert to CNF.

                                          $S \rightarrow AB | T_aB$ \\
                                          $A \rightarrow T_aV_1$ \\
                                          $V_1 \rightarrow T_aT_b$ \\
                                          $B \rightarrow T_bV_2$ \\
                                          $V_2 \rightarrow T_bA$ \\
                                          $T_a \rightarrow a$ \\
                                          $T_b \rightarrow b$
                              \end{enumerate}

                              The language $L(G')$ is

                  \end{enumerate}

      \end{enumerate}

\end{spacing}

\end{document}